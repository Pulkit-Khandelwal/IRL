\documentclass{article}

\usepackage[final]{nips_2016}

\usepackage[utf8]{inputenc} % allow utf-8 input
\usepackage[T1]{fontenc}    % use 8-bit T1 fonts
\usepackage{url}            % simple URL typesetting
\usepackage{booktabs}       % professional-quality tables
\usepackage{amsfonts}       % blackboard math symbols
\usepackage{nicefrac}       % compact symbols for 1/2, etc.
\usepackage{microtype}      % microtypography

\usepackage{amssymb}
\usepackage{amsmath}
\usepackage[titletoc,toc,page]{appendix}
\usepackage{verbatim}
\usepackage[colorlinks=true,citecolor=blue]{hyperref}

\newtheorem{theorem}{Theorem}
\newtheorem{property}{Property}

\newcommand{\needcite}{${}^{\text{\color{red}[citation needed]}}$}


\title{Semi-Supervised Learning on Graphs \\
for Inverse Reinforcement Learning}

\author{
Charles Reizine\\
TODO\\
École Nationale des Ponts et Chaussées\\
Paris, France\\
\texttt{charles.reizine@enpc.fr}
\And
Élie Michel\\
Department of Computer Science\\
École normale supérieure\\
Paris, France\\
\texttt{elie.michel@ens.fr}
}

%\date{October 2016}

\begin{document}

\maketitle

\begin{abstract}
This paper stands as a review of the \emph{Inverse Reinforcement Learning} (IRL) and the interest there is of coupling IRL to \emph{Semi Supervised Learning} ( SSL) using graphs. The IRL 's purpose is to use  \emph{trajectories} ( i-e set of successing states) generated by a set of expert supposedly behaving according to a reward function R, to retrieve this reward function and use it to compute a general optimal policy. In this review, we will focus on the setup where it's easy to observe large set of \emph{trajectories}, but where only a few of them can be labelled. After introducing the general setup of IRL and it limitations, we presented the general purpose and interest of the use of graphs for SSL. Finally, we presented the interest of associating IRL to graphs SSL and presented some of the apparent limitations of this process. 
\end{abstract}

\section{Introduction}

The goal of \emph{Inverse Reinforcement Learning} (IRL) is to learn the reward function in a \emph{Markov Decision Process} (MDP) from the demonstrations of expert agents. It has many applications \cite{Kaelbling96} but, as many ML problems, it often suffers from lack of training data as for instance pointed out in \cite{Vasquez14}. \emph{Semi-Supervised Learning} (SSL) is a typical approach to tackle this issue. In this work, we explore the application of SSL methods to the class of IRL problems.

\paragraph{FIXME}
Write a little § about:
Why is it interesting to focus on this?
Which are the challenges?

 - IRL is ill-posed

 - pb of measure of similarity to apply ssl
 
 - we use ssl to differentiate positive from negative examples

 - apply ssl to objects which are trajectories -> similarity hard to design

% Annonce du plan
\paragraph{}
We first review the challenges of IRL in Section~\ref{sec:irl} on one hand and SSL in Section~\ref{sec:ssl} in the other hand. In Section~\ref{sec:combine}, we present the pipeline in which we combine both problems and the main issues we then face. Then Section~\ref{sec:results} presents our application cases and elements of results. Finally, Section~\ref{sec:limitations} reviews the main limitations of our approach suggest some future work.


%%%%%%%%%%%%%%%%%%%%%%%%%%%%%%%%%%%%%%%%%%%%%%%%%%%%%%%%%%%%%%%%%%%%%%%%%%%%%%%
\section{Inverse Reinforcement Learning \label{sec:irl}}

%------------------------------------------------------------------------------
\subsection{Introduction to Reinforcement Learning}

\paragraph{CITEME}
\cite{Littman94}
\cite{Sutton98} -> book about RL

\paragraph{}
The general idea of Reinforcement Learning (RL) is to determine the optimal action policy of a given agent so that it maximizes a \emph{reward} (also called \emph{reinforcement function}) $R$ provided by its interaction with an environment.

\subsubsection{Formalization of the setup}

This setup is formalized around the model of \emph{Markov Decision Process} (MDP), often assigned to Bellman and his work in Dynamic Programing (already mentioned in \cite{Bellman57}), and used by \cite{Watkins89} as a basis to develop the main RL algorithms such as Q-Learning. A finite MDP is defined as a tuple $(S,A,\{P_{sa}\},\gamma,R)$, where

\begin{itemize}
    \item $S$ is a finite set of $N$ \emph{states}
    \item $A=\{a_1,...,a_k\}$ is a set of $k$ \emph{actions}
    \item $P_{sa}$ are the state \emph{transition probabilities} upon taking action $a$ in state $s$
    \item $\gamma$ is the \emph{discount factor} tunning the long term influence of previous reward
    \item $R$ is the \emph{reward function} (bounded)
\end{itemize}

A \emph{policy} is a map $S \rightarrow A$, potentially probabilistic. Its value is defined as the expected \emph{return} of the agent when sampling actions following this policy:

\[
V^{\pi} = \mathbb E[r_0 + \gamma r_1 + \gamma^2 r_2 + \dots|\pi]
\]

where $r_t$ is the reward received at time $t$.

\subsubsection{Missing Reward}

RL has been applied to a wide variety of problems ranging from robotics \cite{Kober13} to economics \cite{Tesauro02} with impressive results, especially in the field of games \cite{Silver16}. But in many real life problems, the main bottleneck to the application of RL is that the hypothesis of a perfectly known reward function $R$ is not satisfied. For instance, the exact reward maximized during car driving \cite{Shalev16} or human walk \cite{Morimoto04} remains unknown.

%------------------------------------------------------------------------------
\subsection{Inverse problem}

Noticing this issue, \cite{Russell98} introduces the notion of IRL. Instead of looking for a policy given a reward, the algorithm is provided with a policy and asked to determine a reward function for which this policy is optimal. In other words, IRL is intended to build a reward that can explain the observed behavior of an agent.

\subsubsection{First formalization}

This problem has remained ill-posed for some time. A first attempt to formalized it is presented in \cite{Ng00}, which notice that the constraint of $\pi$ being optimal can be rewritten as in Property~\ref{prop:irl}.

\begin{property}
\label{prop:irl}
Let  $(S,A,P_{sa},\gamma,R)$ be a MDP where $S$ is a finite space and $\gamma \in (0,1)$. A policy $\pi$ is optimal if and only if, for all $a=a_{2},...,a_{k}$, the reward satisfies :

\[
(P_{\pi(s)} - P_a)(I - \gamma P_{\pi(s)})^{-1} R \succeq 0
\]
\end{property}

\subsubsection{Additional constraints}

\cite{Ng00} then leverage on this to write IRL as a linear program and apply it to a toy problem (5x5 grid world), but highlights that the problem misses some constraint for the trivial solution where the reward is constant not to be a valid output. Heuristic constraints making the reward more discriminative are explored, such as using reward shaping \cite{Ng99} or maximizing

\[
\sum _{s \in S}\left(Q^\pi(s, a_1) - \max_{a \in A \setminus a_1} Q^\pi(s,a)\right)
\]

\paragraph{}
In large or continuous state systems, the reward cannot be built for each state independently so it is modelized as a combination of hand written feature functions, first linearly combined \cite{Abbeel04} and more recently with gaussian mixture of features \cite{Levine11}.

\subsubsection{Limitations of strict IRL}

\cite{Ng00} quickly notice that this approach scales up poorly to more complex problems that don't come with a full policy. It is particularly problematic for large or continuous state spaces for which it is virtually impossible to enumerate the policy for every state\footnote{unless the problem admits some analytic solution, which is barely never the case in actual applications}.

%------------------------------------------------------------------------------
\subsection{IRL from sampled trajectories}

\paragraph{}
The input is hence relaxed from knowing the whole optimal policy to knowing only a set of trajectories sampled using this trajectories. This matches a setup in which one can record from an expert agent (e.g. a human driver for autonomous driving problem) its sequence of actions given a situation, or many situations, but not explicit the full policy.

\paragraph{}
\cite{Ng00} already addresses this issue and more recent work mainly focus on this case.

\subsubsection{Exploration problem}
The main qualitative issue of using only sampled trajectories to determine $R$ is that it requires these trajectories to go at some point through each point of the state space.

\paragraph{}
So one would need either many trajectories or trajectories long enough with respect to the number of states. Not only it is not tractable in the case of large or continuous state spaces, but there is also no garanty that even among many trajectories one would go through each state.

\subsubsection{Reduction of dimensionality}
To tackle the problem of having too many states to explore, the dimensionality of the reward function is dramatically reduced by feature based approximations \cite{Abbeel04, Levine11} where $R$ is a combination of a number of features small with respect to the number of training trajectories. Alternatively, or complementarily, one could use IRL to build $R$ only on an explored subset of the state space and then generalize using \cite{Finn17}.

\subsubsection{Negative examples}
Another way to augment the training data is to consider also \emph{negative examples}, i.e. examples of bad trajectories that a good policy should not follow.

\paragraph{}
As pointed out in \cite{Lee16}, which introduces a very interesting way of doing IRL with negative examples, this is a way to learn properly in regions of the state space with low reward which are thus rarely explored by positive examples.

\paragraph{}
This approach also has the advantage of invalidating the anoying trivial solutions of IRL such as constant reward and hence makes the problem better posed.

%\subsubsection{Different example policies}
%different policies used as positive examples can be used as advisor to find one single best policy

%%%%%%%%%%%%%%%%%%%%%%%%%%%%%%%%%%%%%%%%%%%%%%%%%%%%%%%%%%%%%%%%%%%%%%%%%%%%%%%
\section{Semi-Supervised Learning \label{sec:ssl}}

\paragraph{TODO} \emph{This whole section needs a refresh and citations.}

%------------------------------------------------------------------------------
\subsection{Semi-supervised learning in Graphs}

Semi-supervised learning is a class of supervised learning tasks and techniques that also make use of unlabeled data for training. This is useful when a large set of data is available but only a small amount of it has already been labeled. Semi-supervised learning falls between unsupervised learning (without any labeled training data) and supervised learning (with completely labeled training data).

The use of graphs for SSL has become widely used after \cite{Zhu03}

It is a transductive approach. \needcite

The idea is to use the input data to build a graphs and to use this graph to label each of the node. To construction of the graph is based on the notion of similarity between nodes. This represents how close two datas are. The closer they are, the more plausible it would be for them to have an identical label.

On some simple problems, the similarity can be built from the euclidian distance between the input vectors. Nonetheless, on more complex issues, similarities are harder to build. We will discuss the mater of building a relevant similarity for trajectories in the next section.

When similarities have been computed, the graph can be built and we can proceed to SSL (via hard or soft harmonic functions for instance \needcite. This is why we will focus on the construction of a relevant similarity function for trajectories.

%%%%%%%%%%%%%%%%%%%%%%%%%%%%%%%%%%%%%%%%%%%%%%%%%%%%%%%%%%%%%%%%%%%%%%%%%%%%%%%
\section{Mixed pipeline of SSL for IRL problem \label{sec:combine}}

The need for SSL arises as soon as we switch from having only positive examples to having both positive and negative examples. While in the first case we do not need labeling in the sens that all samples have the same (positive) label, adding negative examples requires labeling, and so potentially human intervention.

\paragraph{}
While we could imagine more tied mix of SSL and IRL, we focus here on how to apply SSL to objects which are sampled trajectories whose only some of them are labeled (\emph{good} or \emph{bad}) and apply IRL in a second step on reconstructed labels (see Firgure~\ref{fig:pipeline}). This can be seen as a simplification of the task of estimating the return of a trajectory \needcite [is this done w/ DNNs sometimes?].

\begin{figure}
%\includegraphics{}
\caption{TODO Pipeline using SSL to label positive and negative trajectories for IRL\label{fig:pipeline}}
\end{figure}

\paragraph{TODO} state-of-the-art:
\cite{Valko12} -> ?
\cite{Audiffren15} -> ?
\cite{Finn17} -> slightly different case in which the reward is known for a subset of the state space only

%------------------------------------------------------------------------------
\subsection{Trajectory Similarity}

At the core of SSL methods on graphs is the notion of similarity, which enables one to propagate labels from known points to unknown ones. While measuring similarity between fixed size feature vectors is quite straightforward, at least using euclidian distance as a first approximation, designing a measure of similarity between two trajectories is trickier.

\paragraph{TODO}
what did we explore? at least as a mind exprience

\paragraph{NB} Theoretically, it would be important to take into account the reward when comparing trajectories. For instance, trajectories may be very different but both go through the same set of "important" states (the ones actually shaping the reward). But here we won't consider this because we then aim at performing IRL so we don't know the reward function.

\subsubsection{Similarities in Multi-Arm Bandits trajectories}

There is no notion of order in Multi-Arm Bandits trajectories, so we consider the historigram of chosen arms, i.e. their distribution.

\subsubsection{Policy estimation}

This can be generalized to cases in which we have a number of states small wrt. the length of the trajectories. We estimate the underlying matrix $P$ such that:

\[
P_{a,s} = \mathbb P(\text{chosing action } a | \text{being in state } s)
\]

We then compare (KL-divergence?) matrices $P$ estimated from different trajectories to measure their similarity.

\paragraph{Issue} Some states may not be visited by a given trajectory, or visited a too small number of times compared to the number of possible actions. So we introduce a notion of confidence into estimation for each state.

We compare confident enough action distributions and also the vector of confidence among states.

\subsubsection{Bigger number of states/actions}

How to scale to cases in which the number of choices ($n_S$ states times $n_A$ actions) is big comparend to the length $T$ of the trajectories. Cluster RL states? Learn state representation $\phi$? Wavelets? oO


%%%%%%%%%%%%%%%%%%%%%%%%%%%%%%%%%%%%%%%%%%%%%%%%%%%%%%%%%%%%%%%%%%%%%%%%%%%%%%%
\section{Applications and results \label{sec:results}}

using OpenAI's \verb|gym| API

%------------------------------------------------------------------------------
\subsection{MAB}

\emph{Just a theoretical comment.}

%------------------------------------------------------------------------------
\subsection{Student problem}

%------------------------------------------------------------------------------
\subsection{Gridworld}



%%%%%%%%%%%%%%%%%%%%%%%%%%%%%%%%%%%%%%%%%%%%%%%%%%%%%%%%%%%%%%%%%%%%%%%%%%%%%%%
\section{Limitations and future prospects \label{sec:limitations}}

%------------------------------------------------------------------------------
\subsection{Weak coupling of SSL and IRL}

We just concatenated both steps. No real coupling or collaboration/retroaction.

%------------------------------------------------------------------------------
\subsection{Misc notes}

\subsubsection{Many different ways of being wrong: How to learn bad policies}

\subsubsection{A first toy problem}

A good toy problem to test IRL must have:

\begin{itemize}
    \item A reward function which is basically a trade-off between subtargets that we need to weight
    \item Generating (by traditionnal RL) trajectories with différent parameters for the reward trade-off
    \item Reconstructing as much as possible the original weight of R by IRL
    \item A way of labeling positive and pegative trajectories: equivalent to have a priori estimation of R that is +1 for correct trajectories and -1 for wrong ones. Can we release this to a more general hand-crafted prior R?
\end{itemize}

%------------------------------------------------------------------------------
\subsection{Problem of IRL uselessness}

chicken-egg problem: we learn R from trajectories to then compute a policy that is used to generate trajectories. Why not just learning trajectories from trajectories with some kind of recursive model like RNN? This pipeline is just imposing constraints on what can be learned by enforcing the solution to fit the (learned) RL problem.

One reason can be to get more human control. Or a view. See misc notes in Section~\ref{sec:misc}.

%------------------------------------------------------------------------------
\subsection{TODO: Other limitations?}

There is no need for IRL with the rise of DQN like \cite{Mnih15}

lack of time to read papers and perform experimentations (lets be honest)

%%%%%%%%%%%%%%%%%%%%%%%%%%%%%%%%%%%%%%%%%%%%%%%%%%%%%%%%%%%%%%%%%%%%%%%%%%%%%%%
\section{Conclusion}


%%
%% Written by Elie
\subsection{Written by Elie \label{sec:misc}}

\paragraph{TODO}
\emph{Misc though quickly written down by Élie, to be clarified/reorganized/etc.}

The aim of Inverted Reinforcement Learning (IRL) is to learn $R$ so that it can be used in case the set of actions changes. IRL is not just about \emph{generalizing} $\pi$ but also \emph{porting} it to a different type of agent with an action set $mathcal A' \neq \mathcal A$. %% pas compris du tout ce que tu voulais dire %%

Typically\footnote{Give examples}, the machine has access to different information because it has different sensing and computing capabilities. This is why it is important to recover the \emph{intrinseque} reward and not just mimic the training expert policy.

\paragraph{}
Secondly, IRL can be a way of \textbf{getting control} over an AI. $R$, especially represented as a linear combination of some well-known vectors $\Phi_i$, is a very sparse way of representing the behavior of the agent. It makes it humanly possible to \emph{interprete} (if not \emph{understand}) the behavior of the agent, and eventually to \emph{hack} on it to fit the actual objective better.

\paragraph{}
\emph{To explore:} Beside being a reward function actually used to build an agent, $R$ can also be some kind of \emph{view} on the policy followed by an already trained agent. Picking some very simple $\Phi$ can be a way of comparing two subtargets and hence \emph{read} how an agent, whether it is human or artificial, trades of between these targets.

\paragraph{}
Let's take an extremely simplified example for the sake of illustration. Considering the case of a car driver, and imaginning that we got a way of measuring the subtargets $\Phi_{\text{timing}}$, rewarding for a short origin-to-destination timing, and $\Phi_{\text{security}}$, rewarding for safe driving. Even if there are actually many other subtargets involved in the agent true reward $R$, we will consider a \emph{partial reward} $\hat R$ of the form $\hat R = w_1 \Phi_{\text{security}}$



\bibliographystyle{alpha}
\bibliography{bibliography}



\begin{appendices}
\section{Tasks}

\begin{enumerate}
    \item Port Soft-HFS to Python or Q-learning to MATLAB -> Élie
    \item Biblio/State-of-the-art intro paragraph about SSL in graphs -> Charles
    \item Biblio/State-of-the-art intro paragraph about IRL -> Charles
    \item MAB theoretical comments (1 paragraph) -> Élie
    \item generate trajectories for student problem -> Élie/Charles
    \item apply HFS to student problem -> (w/ and w/o confidence vector, and other similarity variances) -> Élie/Charles
    \item Discuss about IRL with negative labeling (biblio and test) -> Charles
    \item apply IRL to stduent problem (if enough time)
    \item Gridworld
    \item list limitations -> both
    \item clean up report -> both
\end{enumerate}


\end{appendices}

\end{document}
